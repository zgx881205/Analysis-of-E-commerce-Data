\documentclass{article}

%
% 引入模板的style文件
%
\usepackage{homework}
\usepackage{xeCJK} %中文包,正常显示中文
\usepackage{ctex} %支持中文,不加这个包,参考文献几个大字显示为英文且出警告
\usepackage{url}
%
% 封面
%

\title{
	\includegraphics[scale = 0.45]{images/title/ucas-logo1.png}\\
    \vspace{1in}
    \textmd{\textbf{\hmwkClass\ \hmwkTitle}}\\
    \textmd{\textbf{\hmwkSubTitle}}\\
    \normalsize\vspace{0.1in}\small{\hmwkCompleteTime }\\
    \vspace{0.1in}\large{\textit{\hmwkClassInstructor\ }}\\
    \vspace{3in}
}

\author{\hmwkAuthorName \\ 
	\hmwkAuthorStuID}
\date{}

\renewcommand{\part}[1]{\textbf{\large Part \Alph{partCounter}}\stepcounter{partCounter}\\}


%
% 正文部分
%
\begin{document}


\maketitle


%\include{chapters/ch01}


\pagebreak

\begin{homeworkProblem}
\textbf{1.	选择的爬虫工具是什么?该工具具有什么特点?}\\
	\textbf{Solution:}\\
	{\color{blue}空
		
	}
\end{homeworkProblem}


\begin{homeworkProblem}
	\textbf{2.	爬虫工具环境部署成功截图。}\\
	\textbf{Solution:}\\
	{\color{blue}空}
	\begin{figure}[H]  % 这里记得用[H]
		\centering
		\includegraphics[width=0.7\linewidth]{images/Fig1}
		\caption{Scrapy环境部署成功}
		\label{fig:ucas-logo}
	\end{figure}
	
\end{homeworkProblem}


\pagebreak

\begin{homeworkProblem}
\textbf{3.	爬虫任务选题
	\begin{itemize}
	\item{选题1:爬取豆瓣电影Top250页面~\footnote{\url{https://movie.douban.com/top250}},获取每部电影的序号、片名、导演、编剧、主演、类型、制作国家/地区、语言、上映日期、片长、又名、豆瓣评分和剧情简介等内容,将数据存入本地txt或者xlsx文件。}
	\item{选题2:爬取南京热门旅游景点的点评数据~\footnote{\url{https://www.mafengwo.cn/jd/10684/gonglve.html}},可以选择其中一个景点,爬取景点的所有文本点评数据(图片不用爬取)。}
    \end{itemize}
}
\textbf{Solution:}\\
{\color{blue}

(1) 打开url并返回BeautifulSop对象: 
\begin{figure}[H]  % 这里记得用[H]
	\centering
	\includegraphics[width=0.75\linewidth]{images/Fig2}
	\caption{返回BeautifulSop对象}
	\label{fig:ucas-logo}
\end{figure}

(2) 解析并获取目标对象: 
\begin{figure}[H]  % 这里记得用[H]
	\centering
	\includegraphics[width=0.75\linewidth]{images/Fig3}
	\caption{解析并获取目标对象}
	\label{fig:ucas-logo}
\end{figure}


(3) 爬虫数据存为csv格式文件 :
	\begin{figure}[H]  % 这里记得用[H]
		\centering
		\includegraphics[width=0.7\linewidth]{images/Fig4}
		\caption{爬虫结果存储}
		\label{fig:ucas-logo}
	\end{figure}

(4)运行执行函数: 
	\begin{figure}[H]  % 这里记得用[H]
	\centering
	\includegraphics[width=0.7\linewidth]{images/Fig5}
	\caption{爬虫执行的主函数}
	\label{fig:ucas-logo}
\end{figure}


(5)爬虫结果展示:
	\begin{figure}[H]  % 这里记得用[H]
	\centering
	\includegraphics[width=1\linewidth]{images/Fig6}
	\caption{CSV保存的爬虫数据}
	\label{fig:ucas-logo}
\end{figure}
}


\end{homeworkProblem}
\pagebreak

\begin{homeworkProblem}
\textbf{4.附加题:在上述的爬虫程序中将获取的数据直接传入数据库,选择的什么数据库,导入数据的代码的截图和最终数据查询的截图。}\\
\textbf{Solution:}\\
{\color{blue}
空
}
\end{homeworkProblem}





% 引用文献
%\bibliographystyle{unsrt}  % unsrt:根据引用顺序编号
%\bibliography{refs}


\end{document}
