\documentclass{article}

%
% 引入模板的style文件
%
\usepackage{homework}
\usepackage{xeCJK} %中文包,正常显示中文
\usepackage{ctex} %支持中文,不加这个包,参考文献几个大字显示为英文且出警告

%
% 封面
%

\title{
	\includegraphics[scale = 0.45]{images/title/ucas-logo1.png}\\
    \vspace{1in}
    \textmd{\textbf{\hmwkClass\ \hmwkTitle}}\\
    \textmd{\textbf{\hmwkSubTitle}}\\
    \normalsize\vspace{0.1in}\small{\hmwkCompleteTime }\\
    \vspace{0.1in}\large{\textit{\hmwkClassInstructor\ }}\\
    \vspace{3in}
}

\author{\hmwkAuthorName \\ 
	\hmwkAuthorStuID}
\date{}

\renewcommand{\part}[1]{\textbf{\large Part \Alph{partCounter}}\stepcounter{partCounter}\\}


%
% 正文部分
%
\begin{document}


\maketitle


%\include{chapters/ch01}


\pagebreak

\begin{homeworkProblem}
\textbf{1.电子商务名词解释:B2B,B2C ,C2B ,C2C ,O2O ,并结合国内外知名电子商务平台案例进行说明。}\\
	{\color{blue}\textbf{答:}B2B:\\
	\qquad B2C:\\
\qquad	C2B:\\
\qquad	C2C:\\
\qquad	O2O:\\
	}
\end{homeworkProblem}


\begin{homeworkProblem}
	\textbf{2.分析大数据6V的特点,如果是参考已有的文章或者网址,请在参考文献部分标出~\cite{zhu}。}\\
{\color{blue}\textbf{答:}大数据的特征可以总结为“6V”,第一,数据量(Volume)大。比如我们在淘宝中的每一次搜索,都会被收集和记录,企业再根据我们的搜索来描绘出用户的肖像,从而精准定位目标客户,推送出客户感兴趣的产品。我国人口众多,形成的数据量自然也很庞大。第二,速度(Velocity)快。速度快包括两方面,流入和流出。从流入的角度来说,每天都有不断流入的动态数据。尤其是在网络时代,我们会留下许多“痕迹”,这些都是累计的数据。由于流入速度快,自然流出——也就是数据处理的速度也要快。

}
	
\end{homeworkProblem}


\pagebreak

\begin{homeworkProblem}
\textbf{3.常用的网络爬虫工具,以及各个工具的特点分析。}\\
{\color{blue}\textbf{答:}

}


\end{homeworkProblem}




% 引用文献
\bibliographystyle{unsrt}  % unsrt:根据引用顺序编号
\bibliography{refs}


\end{document}
